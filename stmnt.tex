\documentclass[12pt]{article}

\usepackage{xcolor}
\usepackage[utf8x]{inputenc}
\usepackage[colorlinks, allcolors=blue]{hyperref}
\usepackage{parskip}
\usepackage{graphicx}
\usepackage{tikz}
\usepackage{wrapfig}
\usepackage{amsfonts}

\begin{document}
\begin{titlepage}
  \centering
  \vspace*{2cm}
  {\Large Analiza Algoritmilor}

  {\Large Tema 4 - Implementarea unei Reduceri Polinomiale}
  \vspace{2cm}

  {\large Termen de predare: {\bf 5 Ianuarie 2016} (100\% punctaj)}

  {\large \hspace{4.2cm} {\bf 12 Ianuarie 2016} (60\% punctaj)}
  \vspace{1cm}

  Ultima actualizare: \color{red} \emph{20.12.2015}

  \vfill
\end{titlepage}

\section{Introducere}
O \emph{reducere Turing} $A \leq_T B$, unde $A$ și $B$ sunt probleme,
ne spune că $B$ este \emph{cel puțin la fel de grea} ca $A$, sau, cu alte
cuvinte, dacă putem rezolva $B$ atunci putem rezolva și $A$ folosind o
transformare \emph{calculabiă} $T$.

O \emph{reducere polinomială}\cite{1} $A \leq_p B$, adaugă o
constrângere la definiția de mai sus: transformarea $T$ a unei instanțe a
problemei $A$ într-o instanță a problemei $B$ poate fi calculată în timp
polinomial ($\exists c \in \mathbb{N}, T = O(n^c)$).

Ne propunem ilustrarea unei astfel de reduceri polinomiale prin implementarea
transformării $T$, adică proiectarea și implementarea unui algoritm care
transformă instanța $in_A$ a problemei $A$, într-o instanță $in_B = T(in_A)$ a
problemei $B$, a.î. $A(in_A) = B(in_B)$, pentru orice $in_A$.

{\bf Notă}: Ultima egalitate din paragraful de mai sus este ceea ce face ca
transformarea să fie {\bf corectă}.

\section{Hamiltonian Path \& SAT}
Un \emph{lanț hamiltonian}\cite{2} într-un graf neorientat $G = (V, E)$ este un
lanț care vizitează fiecare nod \emph{exact} o dată.
Formal, un lanț hamiltonian poate fi exprimat ca o permutare $\pi$ a mulțimii
$\{1, 2, ..., n\}$, a.î:

\begin{itemize}
  \setlength{\itemsep}{0em}
  \item $\pi(i) = j$ $\iff$ al $i$-lea nod vizitat este nodul $j$
  \item $(\pi(i), \pi(i + 1)) \in E \mbox{ for } i = 1,...,n-1$
\end{itemize}

Problema de decizie {\bf HP} se enunță astfel:

\paragraph{}
\emph{Are graful $G$ un lanț hamiltonian}?\vspace{1em}

Reamintim problema de decizie {\bf SAT}\cite{3}:

\paragraph{}
\emph{Dându-se o expresie booleană $\varphi$, există o interpretare $I$ astfel
încât $I \models \varphi$?}\vspace{1em}

\section{Cerință}
Se cere implementarea reducerii {$HP \leq_p SAT$} într-un limbaj de
programare \emph{la alegere}.

Programul va primi ca input un graf neorientat și va trebui să returneze
expresia booleană rezultată ca urmare a aplicării unei tehnici de reducere
{\bf corectă}.

Alături de codul sursă, va fi necesară includerea unui \emph{Makefile} cu
următoarele target-uri:

\begin{itemize}
  \setlength{\itemsep}{0em}
  \item {\bf build}: compilează codul sursă (dacă este cazul)
  \item {\bf run}: rulează programul
  \item {\bf clean}: șterge toate fișierele generate de target-urile
    anterioare, cu excepția celui de output.
\end{itemize}

{\bf Notă}: \emph{make build}, \emph{make run}, \emph{make clean} vor trebui
să fie comenzi valide din root-ul arhivei trimise.

\subsection{Input}
Fișierul de intrare va fi {\bf test.in}.

Pe prima linie se vor afla 2 numere, $V, E$, reprezentând numărul de noduri
din graf, respectiv numărul de muchii ale grafului.
Pe fiecare din următoarele $E$ linii se va afla câte o pereche de forma
$(u, v)$, $1 \le u, v \le V$, cu semnificația \emph{există muchie între nodul
$u$ și nodul $v$}.

\begin{wrapfigure}{r}{0.5\textwidth}
  \raggedright
  % Draw graph
  \tikzstyle{vertex}=[circle,fill=black!25,minimum size=20pt,inner sep=0pt]
  \tikzstyle{edge} = [draw,thick,-]

  \begin{tikzpicture}[remember picture, xscale=1.8]
    % Draw nodes
    \foreach \pos/\name in {{(0,0)/1}, {(1, 1)/2}, {(2, 1)/3},
                            {(3,0)/4}, {(1,-1)/5}}
      \node[vertex] (\name) at \pos {$\name$};

    % Draw edges
    \foreach \source/\dest in {1/5, 5/4, 5/3, 3/2}
      \path[edge] (\source) -- (\dest);
  \end{tikzpicture}
\end{wrapfigure}

\paragraph{Exemplu}\mbox{}\par
5 4\\
1 5\\
5 4\\
5 3\\
3 2

\subsection{Output}
Fișierul de ieșire va fi {\bf test.out}.

Outputul constă într-o singură linie pe care se va afla o expresie booleană.
Ca nume de variabile se vor folosi {\bf xk}, $k = 1..V^2$, unde $V$
este numărul de noduri din graf, cu semnificația: \emph{pe poziția $k/V + 1$ în
HP se află nodul $k\%V + 1$}.
Pentru disjuncție se va folosi caracterul {\bf V}, pentru conjuncție
{\bf $\wedge$} (shift - 6), iar pentru negație {\bf $\sim$} (tilda).
Spațiile vor fi ignorate.

{\bf Notă}: Deoarece nu definim precedența celor 3 operatori,
se vor folosi paranteze rotunde oriunde există ambiguități. Spre exemplu,
expresia {\bf x1$\wedge$x2$\vee$x3} nu este validă. În schimb,
{\bf (x1$\wedge$x2)$\vee$x3}, {\bf x1$\wedge$(x2$\vee$x3)},
{\bf x1$\vee$x2$\vee$x3$\vee$x4} și {\bf x1$\vee$$\sim$x2} sunt expresii valide.

\section{Punctaj}
Tema valorează {\bf 0.5 puncte} din nota finală. Testarea va fi automată.

\section{Referințe}
\renewcommand{\section}[2]{}%
\begin{thebibliography}{9}
  \bibitem[1]{1}
    Polynomial-time reduction\newline
    \url{https://en.wikipedia.org/wiki/Polynomial-time_reduction}
  \bibitem[2]{2}
    Hamiltonian path\newline
    \url{https://en.wikipedia.org/wiki/Hamiltonian_path}
  \bibitem[3]{3}
    Boolean satisfiability problem\newline
    \url{https://en.wikipedia.org/wiki/Boolean_satisfiability_problem}
\end{thebibliography}

\end{document}
